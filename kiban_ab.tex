\documentclass[11pt,a4paper,twoside]{jarticle}
%==== 科研費LaTeX =============================================
%	2013(H25)年度 基盤研究(A,B)(一般)
%============================================================
% 2007-08-19: Taku Yamanaka (JSPS Research Center for Science Systems / Osaka Univ.)
%		Switched to kakenhi3.sty.
% 2007-09-18: Taku: Do not show the budget table beyond the final year.
%============================================================
\input{forms/form00_header}
% user01_header
%=== 様式のファイルの形式の指定 =====================
%   epsではなく、PDF の様式を読み込む場合は、次の行の頭の%を消してください。
%\setboolean{usePDFform}{true}
%=======================================

%=== 予算の表の印刷 =====================
% 予算の集計の表を出すためには、次の行の頭の%を消してください。
%\setboolean{BudgetSummary}{true}
%=================================

% === 一部のページだけタイプセット ==============
% New in 2009 fall version!
% 選んだページだけタイプセットするには、次の例の頭の%を消し、並べてください。
% 複数のページを選ぶこともできます。
% 提出前には、必ず全てコメントアウト(頭に%をつける)してください。
%ーーーーーーーーーーーーーーーーーーーーーーーーーーーーーーーーー
%\KLTypesetPage{1}			% p.1 (or p.1を含む連続したページ),
%\KLTypesetPage{3}			% p.3 (or p.3を含む連続したページ),
%\KLTypesetPagesInRange{5}{6}	% p.5 ~ p.6,
%\KLTypesetPagesInRange{8}{10}	% and p.8 ~ p.10
%=================================

% ===== my favorite packages ====================================
% ここに、自分の使いたいパッケージを宣言して下さい。
\usepackage{wrapfig}
% \usepackage{amssymb}
%\usepackage{mb}
%\DeclareGraphicsRule{.tif}{png}{.png}{`convert #1 `dirname #1`/`basename #1 .tif`.png}
%==========================================================

\newcommand{\KLShouKeiLine}[1]{\cline{#1}}
%もし、小計の上の線を取れと事務に言われたら、
%「そのようなことは、記入要項に書かれていないし、学振はそのようなことは気にしていない。」と
% 突っぱねる。
% それでもなお消せと理不尽なことを言われたら、次の行の 最初の「%」を消す。	
%\renewcommand{\KLShouKeiLine}[1]{}

\newcommand{\KLBudgetTableFontSize}{small}	% 予算の表のフォントの大きさ: small, footnotesize

% ===== my personal definitions ==================================
% ここに、自分のよく使う記号などを定義して下さい。
\newcommand{\klpionn}{K_L \to \pi^0 \nu \overline{\nu}}
\newcommand{\kppipnn}{K^+ \to \pi^+ \nu \overline{\nu}}


\input{forms/hook3} % for future maintenance
% ===== Global definitions for the Kakenhi form ======================
% 基本情報
%
%------ 研究種別 ----------------------------------------------
\newcommand{\研究種別}{B}	% A or B or C

%------ 研究課題名  -------------------------------------------
\newcommand{\研究課題名}{Co-Creativeソフトウェア開発手法のPBL型教育}

%----- 研究機関名と研究代表者の氏名-----------------------
\newcommand{\研究機関名}{産業技術大学院大学}
\newcommand{\研究代表者氏名}{中鉢 欣秀}
\newcommand{\研究代表者氏名ふりがな}{ちゅうばち よしひで}
\newcommand{\me}{\underline{\underline{Y.~Chubachi}}}

%---- 本研究への研究代表者のエフォート(%)
\newcommand{\本応募effort}{\KLEffort{18}}	% 半角数字のみ

%---- 研究期間の最終年度 ----------------
\newcommand{\研究期間の最終元号年度}{27}	%平成で、半角数字のみ
%=========================================================

\input{forms/form02_2013_header}	% <<< New year ?
\input{forms/form03_header}
\input{forms/form05_kiban_ab_header}
\input{forms/form07_header}
%============================================================
%endPrelude

\begin{document}
\input{forms/hook5} % for future maintenance
%============================================================
%     User Inputs
%============================================================

%form: kiban_ab_form_01-02.tex ; user: kiban_ab_01-02_purpose.tex
%========== S-1-7 基盤研究(A,B)(一般) =========
%===== p. 01-02 研究目的 =============
\section{研究目的}
%watermark: w02_purpose_AB
\newcommand{\研究目的概要}{%
%begin  研究目的概要===================
	本研究の真の目的は、一言で言えば、象の卵を見つけるという
         子供の頃からの夢をかなえることである。
    
    そんなことはできるわけない。きっとできない。EGitインストール。もう一度、コミットのテスト。
    Co-Creative Software Development
%end  研究目的概要 ====================
}

\newcommand{\研究目的}{%
%begin  研究目的===================
         \begin{wrapfigure}{r}{5cm}
         		\begin{center}
		         \includegraphics[width=5cm]{figs/seagull2.eps}
		         \caption{カモメ}
		         \label{fig:seagull}
	         \end{center}
         \end{wrapfigure}
	本研究の目的は、象の卵の殻について、生物、化学、物理、工学などの
	方面から多角的に調べることである。
	象の卵の殻は、80kgを超える体重の子象と、
	その栄養源である卵黄の大きな質量を支えるだけではなく、
	卵を暖める親の象の体重も支える必要がある。
	このため、象の卵の殻は、体重の軽い鳥類(図\ref{fig:seagull})の卵の殻とは本質的に異なる構造を持っていると
	考えられる。
	また、象の卵の殻の仕組みが解明されれば、
	\begin{itemize}
		\item 象の生態の解明、恐竜の卵の構造の理解(生物学)、
		\item 殻の化学生成反応の解明(化学)、
		\item 殻の原子レベルでの構造とC$_{60}$やナノクラスターとの関連の研究(物理)、
		\item 人工的に象の殻を作り、車の車体などに応用できる(工学)
	\end{itemize}
	など、科学、社会への影響は計り知れない。

         \begin{wraptable}{r}{0.6\linewidth}
         		\caption{各種動物の、足一本にかかる平均加重}
		\label{tab:load}
         		\begin{tabular}{lrrr}
			\hline
			動物 & 体重 & 足の本数 & 加重(kg/足)\\
			\hline
			ジョロウグモ	& 20mg	& 8	& 2.5mg \\
			象 			& 5t & 4	& 1.3t \\
			人間 			& 60kg	& 2	& 30kg\\
			フラミンゴ	& 10kg	& 1	& 10kg\\
			キングコブラ	& 7kg	& 0	& $\infty$\\
			\hline
		\end{tabular}
         \end{wraptable}

         紀元前に、アルキメデス(\('A\rho\chi i\mu\acute{\eta}\delta\eta\mbox{\c{c}}\))は
	象の卵の形を円筒座標表示で
         \[r(z) = 0.5\sqrt{1-(e^z-2)^2}\]
         で近似し、その体積を求めようとしたが、当時はまだ
         \begin{equation}
	         V  = \pi \int_0^{\ln 3} r^2(z) dz\\
         \end{equation}
         の計算が難しくあきらめていた。
         しかしある日、好物の温泉卵を作ろうとして鶏の卵を持って入浴している最中に、
         風呂からあふれるお湯を見て、象の卵の体積を測定する方法を思いついたと言われる。
 
	さて、象の卵の殻の強度については、すでに19世紀初めにロシアのキーファ・モキエーイチが
	考察していると、ゴーゴリが紹介している
	\cite{gogori}。
	しかし、この斬新で自由な発想にもとづく科学的考察に対し、
	トルストイは果敢にも、
	そういう考察がいかに論理的であろうとそれ自体間違っていて無駄である、
	と厳しく批判している
	\cite{torusutoi}。
	これは、既成概念にとらわれた、科学に対する挑戦ともとれるが、
	まだ進化論が現代の米国のように広く信じられていなかった
	帝政ロシアの時代にあっては、
	(進化論が米国で広く信じられているかどうかは、読み手の、文の解釈の仕方による)
	トルストイでさえも象の卵に対してこのような考えを
	持たざるを得なかったのは、理解できない事ではないと言わざるを得ないであろう。
	
	日本でも昔はナウマン象が生息しており、
	その名残は各地に残っている。
	例えば逢坂北部のある終点駅の駅前では、
	毎年年末になると図\ref{fig:egg_R}, \ref{fig:egg_L}に示すように
	象の卵の像のまわりを電飾するしきたりが残っている。
	(少し寄り目にし、右目で左の図、左目で右の図を見てください。
	なお、このように図や表を横に並べる方が、{\tt wrapfigure}を用いるより位置の調整が楽です。)
        \begin{figure}[h]
         	\begin{minipage}[t]{0.49\linewidth}
			\includegraphics[width=\linewidth]{figs/egg_R.eps}
			\caption{右目用}
			\label{fig:egg_R}
		\end{minipage}
		\hspace{0.01\linewidth}
		\begin{minipage}[t]{0.49\linewidth}
			\includegraphics[width=\linewidth]{figs/egg_L.eps}
			\caption{左目用}
			\label{fig:egg_L}
		\end{minipage}
         \end{figure}

	また、寺村輝夫の研究\cite{teramura}によれば、昔、
	王子の誕生を祝って国民全員に卵焼きを提供すべく、
	軍隊を動員して象の卵を探させた王がいた。
	このときは孵化直後の子象は見つかったが、それが入っていた殻の発見には至っていない。
	人の家の裏庭の犬小屋を衛星写真で調べることさえもできなかった時代とあっては、
	この失敗も無理からぬことである。
	
	しかし今や、進化論は確立し、遺伝子の解析による派生の系統解析や
	犯人の特定ができる時代である。
	また、土を掘り返すことを基本としていた考古学でも、
	宇宙からナスカの近くに新たな地上絵を発見する時代である。
	このように、
	現代の科学技術を駆使すれば、マクロな広範囲に渡る精細な探索と、
	ミクロな遺伝子からの解析は可能であり、
	象の卵を世界に先駆けて発見することは、科学技術立国としての日本に課せられた使命でもある
	と言っても過言ではない。
	
	\vspace{1cm}
	\begin{thebibliography}{99}
		\bibitem{gogori} ゴーゴリ、「死せる魂」(1841).
		\bibitem{torusutoi} トルストイ、「人生論」(1886).
		\bibitem{teramura} 寺村輝夫、「ぼくは王様 - ぞうのたまごのたまごやき」.
	\end{thebibliography}
%end  研究目的 ====================
}

%====================================
%form: kiban_ab_form_03-05.tex ; user: kiban_ab_03-05_plan.tex
%========== S-1-7 基盤研究(A,B)(一般) =========
%===== p. 03-05 研究計画・方法 =============
\section{研究計画・方法}
%watermark: w08_plan_AB
\newcommand{\研究計画と方法概要}{%
%begin  研究計画と方法概要===================
	ぞうの卵を探すために、世界中を旅する。
	これも子供の頃からの夢であった。
%end  研究計画と方法概要 ====================
}

\newcommand{\研究計画}{%
%begin  研究計画===================
	初年度は、まず世界の動物園を巡り、
	研究業績 \KLcite{pub:theoegg}に可能性が示されたように
	象舍に卵が隠されていないか、探す。

	2年目はアフリカに行き、空と地上から象の卵を探す。
	アフリカ象は気性が荒いが、サバンナの方がジャングルよりも見通しが効くので、
	インドよりもアフリカを先に探索する。

	3年目は、インドとタイに行き、ジャングルに隠されている卵を探す。
	ジャングルの場合は空からは探しにくいが、象使いも多く、象の背中に乗って
	象の視点から探索することができる。
	さらに、気だての優しいインド象ならば
	卵の在処を教えてくれる可能性もある。
	
	\input{blahblah}  % << only for demonstration. Please delete it or comment it out.
	\input{blahblah}  % << only for demonstration. Please delete it or comment it out.
		
%end  研究計画 ====================
}

%form: kiban_ab_form_06.tex ; user: kiban_ab_06_preparation_final_year.tex
%========== S-1-7 基盤研究(A,B)(一般) =========
%===== p. 06 準備状況等、最終年度の応募 =============
\section{準備状況等、最終年度の応募}
\subsection{準備状況等}
\newcommand{\準備状況等}{%
%begin  準備状況等 ===================
	象の卵について、文献調査を行っている。
	Dr.~Seussは"Horton Hatches the Egg"という論文を1940年に発表している。
	また最近では2004年に、南カルフォニア大のSam Yousefianの率いる研究チームが "The Elephant's Egg"という記録映画を発表している。\\
	({\tt http://www.bangbang.tv/syelephant.html})

	我々はさらに一歩進め、研究の経過を紹介する「threeD」の
	ドキュメンタリー映画を作って全国でロードショーを行う。
%end  準備状況等 ====================
}

\subsection{研究計画最終年度の応募}
\newcommand{\研究計画最終年度の応募の研究種目名}{%
%begin  研究種目名 ===================
         基盤研究A
%end  研究種目名 ====================
}

\newcommand{\研究計画最終年度の応募の審査区分}{%
%begin  審査区分 ===================
	123
%end  審査区分 ====================
}

\newcommand{\研究計画最終年度の応募の課題番号}{%
%begin  課題番号 ===================
12345678	%半角(英数字)
%end  課題番号 ====================
}

\newcommand{\研究計画最終年度の応募の研究課題名}{%
%begin  研究課題名 ===================
	シロナガスクジラの卵の殻はなぜ見つからないのか
%end  研究課題名 ====================
}

\newcommand{\研究計画最終年度の応募の研究期間初年度}{%
%begin  研究期間初年度 ===================
	15
%end  研究期間初年度 ====================
}

\newcommand{\研究計画最終年度の応募の計画と成果}{%
%begin  特別推進研究又は基盤研究による研究計画及び研究成果 ===================
	研究課題の通り、シロナガスクジラの卵は見つけられなかった。
%end  特別推進研究又は基盤研究による研究計画及び研究成果 ====================
}

\newcommand{\研究計画最終年度の応募の理由}{%
%begin  研究計画最終年度前年度の応募をする理由 ===================
	さっさと次の研究に移りたいので。
%end  研究計画最終年度前年度の応募をする理由 ====================
}

%====== end of page =====================================
%form: kiban_ab_form_07-09.tex ; user: kiban_ab_07-09_publications.tex
%========== S-1-7 基盤研究(A,B)(一般) =========
%===== p. 07-09 研究業績 =============
\section{研究業績}
%watermark: w14_pub_AB
% 2012-09-01 Taku
\newcommand{\年と名前と研究業績}{%
%begin  研究業績 ===================
		% 2013年度から始まった2カラムのtabularです。
%ーーーーーーーーーーーーーーーーーーーーーーーーーーーーーーーーーーーーーーーーーーーー
%		\KLcite{pub:theoegg} のようにして業績番号を文中に入れられます。
%ーーーーーーーーーーーーーーーーーーーーーーーーーーーーーーーーーーーーーーーーーーーー
	2012 {\small 以降} \\
		殻十象
		& \KLbibitem \label{pub:theoegg} "Theory of Elephant Eggs", 
				*\underline{Juzo Kara}, \me, 
				\settensen \Euc{Renkei Musashino} {\it et al.},
				Phys.\ Rev.\ Lett. {\bf 800}, 800-804 (2012). \\
		%............		
		殻十象
		& \KLbibitem \label{pub:theowhale} "Theory of Whale Eggs",
				*\underline{Juzo Kara} {\it et al.},
				Phys.\ Rev.\ Lett. {\bf 800}, 805-808 (2012). \\
	\hline%----------------------------------------------
	
	2011 \\
		安倍公房
		&  \KLbibitem \label{pub:abe} "仔象は死んだ", 
				*\underline{Kobo Abe},
				安部公房全集, {\bf 26}, 100-200, (2011). \\

	\hline%----------------------------------------------
	2010 \\
		Rudyard Kipling
		&  \KLbibitem "The Elephant's Child (象の鼻はなぜ長い)", 
				*\underline{R.~Kipling} and \me,
				Nature, {\bf 999}, 777-779, (2010). \\

	\hline%----------------------------------------------

	2009 \\
		 Walt Disney
		 &  \KLbibitem "Dumbo", 
				*\underline{Walt Disney},
				Disney J., {\bf 314}, 159-265, (2009). \\

	\hline%----------------------------------------------

	2008 \\
		Alan Cooper
		&  \KLbibitem "Egg of Elephant-Bird", 
				*\underline{A.~Cooper},
				Nature, {\bf 409}, 704-707 (2008). \\
     

\input{jack_pub_w_year_2cols}	% <<< only for demonstration.  Please delete it or comment it out.

%end  研究業績  ====================
}

\newcommand{\連携研究者の研究業績}{%
%begin  連携研究者の研究業績 ===================
		% 2カラムのtabularです。
%ーーーーーーーーーーーーーーーーーーーーーーーーーーーーーーーーーーーーーーーーーーーー
%		\KLciteB{pub:theoegg} のようにして業績番号を文中に入れられます。
%ーーーーーーーーーーーーーーーーーーーーーーーーーーーーーーーーーーーーーーーーーーーー
		殻十象 
			& \KLbibitemB "Theory of Elephant Eggs", 
				\settensen *\Euc{Juzo Kara} {\it et al.},
				Phys.\ Rev.\ Lett. {\bf 800}, 800-804 (2005). \\
			& \KLbibitemB "Search for whale eggs", 
				\settensen *\Euc{Juzo Kara}, Anim.\ Rev.\ D, 1956-1960 (1951).\\

		\hline
		
		安倍公房
			& \KLbibitemB "仔象は死んだ", 
				\settensen *\Euc{Kobo Abe},
				安部公房全集, {\bf 26}, 100-200, (2004).\\
		\hline
		Rudyard Kipling
			& \KLbibitemB \label{pub:kipling} "The Elephant's Child (象の鼻はなぜ長い)", 
				\settensen *\Euc{R.~Kipling},
				Nature, {\bf 999}, 777-779, (2003).\\
		\hline
		Alan Cooper
			& \KLbibitemB \label{pub:cooper} "Egg of Elephant-Bird", 
				\settensen *\Euc{A.~Cooper},
				Nature, {\bf 409}, 704-707 (2001).\\
%end  連携研究者の研究業績  ====================
}

%===========================================================
 %=======================================================
%form: kiban_ab_form_10.tex ; user: kiban_ab_10_past_funds.tex
%========== S-1-7 基盤研究(A,B)(一般) =========
%===== p. 10 これまでに受けた研究費とその成果等 =============
\section{これまでに受けた研究費とその成果等}
\newcommand{\これまでに受けた研究費とその成果等}{%
%begin  これまでに受けた研究費とその成果等 ===================
	\begin{itemize}
		\item 基盤研究(A)(一般)、2004-2005年度、「鯨の卵」、研究代表者、1,234千円\\
			地球上で最大の生物、シロナガスクジラの卵の研究した。
			クジラの卵の場合は、高い水圧に耐える必要があるため、
			堅固の構造となっているはずであり、
			これが解明されれば、将来、深海潜水艇への応用も効く。
			しかし、シロナガスクジラの生息範囲が広い、海に潜っている時間が長い、
			生息数も減っている、などの原因により、
			卵を見つけることができなかった。	
					
	\hrulefill %------------------------------- (この上下に空行をいれること)----------
	
		\item 非科学研究補助金、2001年度、
			「マイナスイオンによる地球分裂」、研究分担者、800千円\\
			マイナスイオンを発生する装置の増大に伴い、電荷間の反発力が自己重力
			に打ち勝つ事によって地球が粉々に
			分裂する可能性について、詳細な検討を行った。
			この研究は、地球という惑星を一つ失うことによる占星術への影響を懸念する
			団体から補助金を得て行った。
	\end{itemize}
%end  これまでに受けた研究費とその成果等 ====================
}

%====== end of page =====================================
%form: kiban_ab_form_11.tex ; user: kiban_ab_11_relation.tex
%========== S-1-7 基盤研究(A,B)(一般) =========
%===== p. 11 研究計画と研究進捗評価を受けた研究課題の関連性 =============
\section{研究計画と研究進捗評価を受けた研究課題の関連性}
\newcommand{\研究計画と研究進捗評価を受けた研究課題の関連性}{%
%begin  研究計画と研究進捗評価を受けた研究課題の関連性 ===================
	研究進捗評価を受けた研究課題は、シロナガスクジラの卵の探索である。
	それに対し、本研究は象の卵の探索である。
	シロナガスクジラも象も、ともにほ乳類であるという共通点は持つが、
	生息する所が全く異なる。
%end  研究計画と研究進捗評価を受けた研究課題の関連性 ====================
}

%====== end of page =====================================
%form: kiban_ab_form_12.tex ; user: kiban_ab_12_human_rights_etc.tex
%========== S-1-7 基盤研究(A,B)(一般) =========
%===== p. 12 人権、法令、研究経費の妥当性など =============
\section{人権、法令、研究経費の妥当性など}
\newcommand{\人権の保護及び法令等の遵守への対応}{%
%begin  人権の保護及び法令等の遵守への対応 ===================
	象の卵のES細胞の培養、象のクローンの生成などは行わない。
	象個体を現地から持ち出すことはないので、ワシントン条約ならびに
        生物多様性条約に抵触しない。また、組換え実験は行なわないので、
        カルタヘナ議定書にも抵触しない。
%end  人権の保護及び法令等の遵守への対応 ====================
}

\subsection{研究経費の妥当性・必要性}
\newcommand{\研究経費の妥当性と必要性}{%
%begin  研究経費の妥当性・必要性 ===================
	「研究計画・方法」欄で述べた研究規模、研究体制等を踏まえると、
	次頁以降に記入する研究費は妥当、かつ必要であり、
	積算根拠も妥当である。
%end  研究経費の妥当性・必要性 ====================
}

%====== end of page =====================================
%form: kiban_ab_form_13.tex ; user: kiban_ab_13_materials.tex
%========== S-1-7 基盤研究(A,B)(一般) =========
%===== p. 13 設備備品費、消耗品費の明細 =============
\section{設備備品費、消耗品費の明細}
%				\KLJFY{\1年目J}
\newcommand{\設備備品費1年目}{%
%begin  設備備品費1年目 ===================
	\KLItemNumUnitCostLocation{タケコプター}{2}{123000}{ケニア大学}
	\KLItemNumUnitCostLocation{どこでもドア}{1}{80000}{どこでもよい}
%end  設備備品費1年目 ====================
}

\newcommand{\消耗品費1年目}{%
%begin  消耗品費1年目 ===================
	% 2カラムのtabularです。
	\KLItemCost{タケコプター燃料}{56789}
	\KLItemCost{象の餌代}{10000}
	\KLItemCost{卵切断用鋸}{1000}
%end  消耗品費1年目 ====================
}

\newcommand{\設備備品費2年目}{%
%begin  設備備品費2年目 ===================
	\KLItemNumUnitCostLocation{タケコプター}{2}{123000}{ケニア大学}
	\KLItemNumUnitCostLocation{大型フライパン}{2}{20}{どこでもよい}
%end  設備備品費2年目 ====================
}

\newcommand{\消耗品費2年目}{%
%begin  消耗品費2年目 ===================
	\KLItemCost{タケコプター燃料}{80000}
	\KLItemCost{象の餌代}{20000}
	\KLItemCost{ハードディスク}{2000}
%end  消耗品費2年目 ====================
}

\newcommand{\設備備品費3年目}{%
%begin  設備備品費3年目 ===================
	\KLItemNumUnitCostLocation{タケコプター}{3}{123000}{ケニア大学}
	\KLItemNumUnitCostLocation{大型フライパン}{3}{20}{どこでもよい}
%end  設備備品費3年目 ====================
}

\newcommand{\消耗品費3年目}{%
%begin  消耗品費3年目 ===================
	\KLItemCost{象の餌代}{30000}
	\KLItemCost{ハードディスク}{3000}
%end  消耗品費3年目 ====================
}

\newcommand{\設備備品費4年目}{%
%begin  設備備品費4年目 ===================
	\KLItemNumUnitCostLocation{タケコプター}{4}{123000}{ケニア大学}
	\KLItemNumUnitCostLocation{大型フライパン}{4}{20}{どこでもよい}
%end  設備備品費4年目 ====================
}

\newcommand{\消耗品費4年目}{%
%begin  消耗品費4年目 ===================
	\KLItemCost{象の餌代}{40000}
	\KLItemCost{ハードディスク}{4000}
%end  消耗品費4年目 ====================
}

\newcommand{\設備備品費5年目}{%
%begin  設備備品費5年目 ===================
	\KLItemNumUnitCostLocation{タケコプター}{5}{123000}{ケニア大学}
	\KLItemNumUnitCostLocation{大型フライパン}{5}{20}{どこでもよい}
%end  設備備品費5年目 ====================
}

\newcommand{\消耗品費5年目}{%
%begin  消耗品費5年目 ===================
	\KLItemCost{象の餌代}{50000}
	\KLItemCost{ハードディスク}{5000}
%end  消耗品費5年目 ====================
}

%====== end of page =====================================
%form: kiban_ab_form_14.tex ; user: kiban_ab_14_travels.tex
%========== S-1-7 基盤研究(A,B)(一般) =========
%===== p. 14 旅費等の明細 =============
\section{旅費等の明細}
%	\KLNoMarginMinipage{66}{706}{566}{
%				\KLJFY{\1年目J}
\newcommand{\国内旅費1年目}{%
%begin  国内旅費1年目 ===================
		% 2カラムのtabularです。
		\KLItemCost{探検打合わせ}{150}
		\KLItemCost{象の調査}{120}
%end  国内旅費1年目 ====================
}

\newcommand{\外国旅費1年目}{%
%begin  外国旅費1年目 ===================
		% 2カラムのtabularです。
		\KLItemCost{卵収集}{1500}
		\KLItemCost{象の調査}{1200}
%end  外国旅費1年目 ====================
}

\newcommand{\謝金等1年目}{%
%begin  謝金等1年目 ===================
		% 2カラムのtabularです。
	 	\KLItemCost{パイロット報酬}{3000}
		\KLItemCost{ハンター賃金}{1000}
%end  謝金等1年目 ====================
}

\newcommand{\その他1年目}{%
%begin  その他1年目 ===================
		% 2カラムのtabularです。
	 	\KLItemCost{通信費}{800}
		\KLItemCost{卵運搬費}{4000}
		\KLItemCost{ジープ借料}{4100}
%end  その他1年目 ====================
}

\newcommand{\国内旅費2年目}{%
%begin  国内旅費2年目 ===================
		% 2カラムのtabularです。
		\KLItemCost{探検打合わせ}{250}
		\KLItemCost{象の調査}{220}
%end  国内旅費2年目 ====================
}

\newcommand{\外国旅費2年目}{%
%begin  外国旅費2年目 ===================
		% 2カラムのtabularです。
		\KLItemCost{卵収集}{2500}
		\KLItemCost{象の調査}{2200}
%end  外国旅費2年目 ====================
}

\newcommand{\謝金等2年目}{%
%begin  謝金等2年目 ===================
	 	\KLItemCost{パイロット報酬}{3000}
		\KLItemCost{ハンター賃金}{2000}
%end  謝金等2年目 ====================
}

\newcommand{\その他2年目}{%
%begin  その他2年目 ===================
	 	\KLItemCost{通信費}{800}
		\KLItemCost{卵運搬費}{4000}
		\KLItemCost{ジープ借料}{4200}
%end  その他2年目 ====================
}

\newcommand{\国内旅費3年目}{%
%begin  国内旅費3年目 ===================
		% 2カラムのtabularです。
		\KLItemCost{探検打合わせ}{350}
		\KLItemCost{象の調査}{320}
%end  国内旅費3年目 ====================
}

\newcommand{\外国旅費3年目}{%
%begin  外国旅費3年目 ===================
		% 2カラムのtabularです。
		\KLItemCost{卵収集}{3500}
		\KLItemCost{象の調査}{3200}
%end  外国旅費3年目 ====================
}

\newcommand{\謝金等3年目}{%
%begin  謝金等3年目 ===================
	 	\KLItemCost{パイロット報酬}{3000}
		\KLItemCost{ハンター賃金}{3000}
%end  謝金等3年目 ====================
}

\newcommand{\その他3年目}{%
%begin  その他3年目 ===================
	 	\KLItemCost{通信費}{800}
		\KLItemCost{卵運搬費}{4000}
		\KLItemCost{ジープ借料}{4300}
%end  その他3年目 ====================
}

\newcommand{\国内旅費4年目}{%
%begin  国内旅費4年目 ===================
		% 2カラムのtabularです。
		\KLItemCost{探検打合わせ}{450}
		\KLItemCost{象の調査}{420}
%end  国内旅費4年目 ====================
}

\newcommand{\外国旅費4年目}{%
%begin  外国旅費4年目 ===================
		% 2カラムのtabularです。
		\KLItemCost{卵収集}{4500}
		\KLItemCost{象の調査}{4200}
%end  外国旅費4年目 ====================
}

\newcommand{\謝金等4年目}{%
%begin  謝金等4年目 ===================
	 	\KLItemCost{パイロット報酬}{3000}
		\KLItemCost{ハンター賃金}{4000}
%end  謝金等4年目 ====================
}

\newcommand{\その他4年目}{%
%begin  その他4年目 ===================
	 	\KLItemCost{通信費}{800}
		\KLItemCost{卵保管費}{4400}
%end  その他4年目 ====================
}

\newcommand{\国内旅費5年目}{%
%begin  国内旅費5年目 ===================
		% 2カラムのtabularです。
		\KLItemCost{探検打合わせ}{550}
		\KLItemCost{象の調査}{520}
%end  国内旅費5年目 ====================
}

\newcommand{\外国旅費5年目}{%
%begin  外国旅費5年目 ===================
		% 2カラムのtabularです。
		\KLItemCost{卵収集}{5500}
		\KLItemCost{象の調査}{5200}
%end  外国旅費5年目 ====================
}

\newcommand{\謝金等5年目}{%
%begin  謝金等5年目 ===================
	 	\KLItemCost{パイロット報酬}{3000}
		\KLItemCost{ハンター賃金}{5000}
%end  謝金等5年目 ====================
}

\newcommand{\その他5年目}{%
%begin  その他5年目 ===================
	 	\KLItemCost{通信費}{800}
		\KLItemCost{卵保管費}{4500}
%end  その他5年目 ====================
}

%====== end of page =====================================
%form: kiban_ab_form_15.tex ; user: kiban_ab_15_other_applications.tex
%========== S-1-7 基盤研究(A,B)(一般) =========
%===== p. 15 研究費の応募・受入等の状況・エフォート =============
\section{研究費の応募・受入等の状況・エフォート}
\subsection{応募中の研究費}
\newcommand{\本人の研究経費}{%
%begin  本人の研究経費 ===================
	\KLMyBudget{}{}% 初年度と、期間全体で「本人が」使う額
%	分担者がいない場合は、\KLMyBudget{}{} のように{}の中を空にしてください。金額が自動的に入ります。
%end  本人の研究経費  ====================
}

\newcommand{\応募中の研究費}{%
%begin  応募中の研究費 ===================
		%6カラムのtabular
		% 1:資金制度・研究費名(研究期間・配分機関名)
		% 2:研究課題名(研究代表者氏名)
		% 3:代表/分担
		% 4:初年度の研究経費(期間全体の額)
		% 5:エフォート
		% 6:研究内容の相違点及び他の研究費に加えて本応募研究課題に応募する理由
%		\multicolumn{6}{c}{\dotfill} \\
		
		
%end  応募中の研究費 ====================
}

%====== end of page =====================================
%form: kiban_ab_form_16.tex ; user: kiban_ab_16_other_funds.tex
%========== S-1-7 基盤研究(A,B)(一般) =========
%===== p. 16 受け入れ予定の研究費 =============
\subsection{受け入れ予定の研究費}
%    \KLNoMarginMinipage{\KLLeftEdge}{694}{568}{
\newcommand{\受け入れ予定の研究費}{%
%begin  受け入れ予定の研究費 ===================
%		\multicolumn{6}{c}{\dotfill} \\
%end  受け入れ予定の研究費 ====================
}

%====== end of page =====================================
%\KLCheckPageLimit
%\KLAdvancePages
% hook9 : right before \end{document} ============

%endUserFiles
\input{forms/hook7} % for future maintenance

% kiban_ab_forms
%=======================================
\ifthenelse{\boolean{BudgetSummary}}{
	\KLTypesetPage{13}
	\KLTypesetPage{14}
}{}
	
\ifthenelse{\boolean{BudgetSummary}\OR \boolean{klTypesetPage0}}{
	\input{forms/coverpage}
}{}

\KLInputIfPageInRangeIsSelected{1}{2}{forms/kiban_ab_form_01-02}
\KLInputIfPageInRangeIsSelected{3}{5}{forms/kiban_ab_form_03-05}
\KLInputIfSelected{6}{forms/kiban_ab_form_06}
\KLInputIfPageInRangeIsSelected{7}{9}{forms/kiban_ab_form_07-09}
\KLInputIfSelected{10}{forms/kiban_ab_form_10}
\KLInputIfSelected{11}{forms/kiban_ab_form_11}
\KLInputIfSelected{12}{forms/kiban_ab_form_12}
\KLInputIfSelected{13}{forms/kiban_ab_form_13}
\KLInputIfSelected{14}{forms/kiban_ab_form_14}

\ifthenelse{\boolean{BudgetSummary}}{
	\input{forms/form90_summary}
}{%
}

\KLInputIfSelected{15}{forms/kiban_ab_form_15}
\KLInputIfSelected{16}{forms/kiban_ab_form_16}

%========================================

%endFormatFile

\input{forms/hook9} % for future maintenance
\end{document}
