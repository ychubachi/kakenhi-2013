\documentclass[11pt,a4paper,twoside]{jarticle}
%==== 科研費LaTeX =============================================
%	2013(H25)年度 基盤研究(A,B)(一般)
%============================================================
% 2007-08-19: Taku Yamanaka (JSPS Research Center for Science Systems / Osaka Univ.)
%		Switched to kakenhi3.sty.
% 2007-09-18: Taku: Do not show the budget table beyond the final year.
%============================================================
\input{forms/form00_header}
% user01_header
%=== 様式のファイルの形式の指定 =====================
%   epsではなく、PDF の様式を読み込む場合は、次の行の頭の%を消してください。
%\setboolean{usePDFform}{true}
%=======================================

%=== 予算の表の印刷 =====================
% 予算の集計の表を出すためには、次の行の頭の%を消してください。
%\setboolean{BudgetSummary}{true}
%=================================

% === 一部のページだけタイプセット ==============
% New in 2009 fall version!
% 選んだページだけタイプセットするには、次の例の頭の%を消し、並べてください。
% 複数のページを選ぶこともできます。
% 提出前には、必ず全てコメントアウト(頭に%をつける)してください。
%ーーーーーーーーーーーーーーーーーーーーーーーーーーーーーーーーー
%\KLTypesetPage{1}			% p.1 (or p.1を含む連続したページ),
%\KLTypesetPage{3}			% p.3 (or p.3を含む連続したページ),
%\KLTypesetPagesInRange{5}{6}	% p.5 ~ p.6,
%\KLTypesetPagesInRange{8}{10}	% and p.8 ~ p.10
%=================================

% ===== my favorite packages ====================================
% ここに、自分の使いたいパッケージを宣言して下さい。
\usepackage{wrapfig}
% \usepackage{amssymb}
%\usepackage{mb}
%\DeclareGraphicsRule{.tif}{png}{.png}{`convert #1 `dirname #1`/`basename #1 .tif`.png}
%==========================================================

\newcommand{\KLShouKeiLine}[1]{\cline{#1}}
%もし、小計の上の線を取れと事務に言われたら、
%「そのようなことは、記入要項に書かれていないし、学振はそのようなことは気にしていない。」と
% 突っぱねる。
% それでもなお消せと理不尽なことを言われたら、次の行の 最初の「%」を消す。	
%\renewcommand{\KLShouKeiLine}[1]{}

\newcommand{\KLBudgetTableFontSize}{small}	% 予算の表のフォントの大きさ: small, footnotesize

% ===== my personal definitions ==================================
% ここに、自分のよく使う記号などを定義して下さい。
\newcommand{\klpionn}{K_L \to \pi^0 \nu \overline{\nu}}
\newcommand{\kppipnn}{K^+ \to \pi^+ \nu \overline{\nu}}


\input{forms/hook3} % for future maintenance
% ===== Global definitions for the Kakenhi form ======================
% 基本情報
%
%------ 研究種別 ----------------------------------------------
\newcommand{\研究種別}{A}	% A or B or C

%------ 研究課題名  -------------------------------------------
\newcommand{\研究課題名}{コ・クリエイティブなソフトウェア開発者を育成するPBL型教育}

%----- 研究機関名と研究代表者の氏名-----------------------
\newcommand{\研究機関名}{産業技術大学院大学}
\newcommand{\研究代表者氏名}{中鉢 欣秀}
\newcommand{\研究代表者氏名ふりがな}{ちゅうばち よしひで}
\newcommand{\me}{\underline{\underline{中鉢 欣秀}}}
\newcommand{\meen}{\underline{\underline{Y.~Chubachi}}}

%---- 本研究への研究代表者のエフォート(%)
\newcommand{\本応募effort}{\KLEffort{18}}	% 半角数字のみ

%---- 研究期間の最終年度 ----------------
\newcommand{\研究期間の最終元号年度}{27}	%平成で、半角数字のみ
%=========================================================

\input{forms/form02_2013_header}	% <<< New year ?
\input{forms/form03_header}
\input{forms/form05_kiban_ab_header}
\input{forms/form07_header}
%============================================================
%endPrelude

\begin{document}
\input{forms/hook5} % for future maintenance
%============================================================
%     User Inputs
%============================================================

%form: kiban_ab_form_01-02.tex ; user: kiban_ab_01-02_purpose.tex
%========== S-1-7 基盤研究(A,B)(一般) =========
%===== p. 01-02 研究目的 =============
\section{研究目的}
%watermark: w02_purpose_AB
\newcommand{\研究目的概要}{%
%begin  研究目的概要===================
	コ・クリエイティブなソフトウェア開発方法論とは,ソフトウェア開発者がマーケットとの直接的な対話を通して
	市場でマネタイズできるソフトウェアサービスを開発するための新しい開発プロセスである.本研究では,この
	開発プロセスをプロジェクト型学習(PBL)により教育するための教材および教授法を開発することを目的とする.
	
	本研究者らが行なってきたPBLによるソフトウェア技術者の教育実績を踏まえ,指導者のためのガイドライン,
	PBLを支援するためのインフラストラクチャー,その他必要な教材の整備を行い,この教育手法を確立させ,
	次世代型のソフトウェア開発者育成法として普及を図る.
%end  研究目的概要 ====================
}

\newcommand{\研究目的}{%
%begin  研究目的===================
	
	\begin{flushleft}
		■コ・クリエイティブなソフトウェア開発
	\end{flushleft}

    「コ・クリエイション(co-creation)」とは,マーケティング分野の用語であり,
    商品やサービスの開発にあたり企業が顧客を巻き込むことでよりよいものを創りだすことを指す.
    コ・クリエイションの最近の事例としては,Starbucks,Dellなどが顧客のアイディア
    をソーシャルメディアにより収集し,自社のサービス改善につなげていることが報告されている\cite{wired}.

    一方で,ソフトウェア開発においては,従来からLinuxを代表とするオープンソース型の
    ソフトウェア開発のスタイルに見られるように,コ・クリエイティブにソフトウェアプロダクトを開発する事例が数多くある.
    ソフトウェアを開発するデベロッパーとユーザとの間に垣根がなく,ユーザは自ら必要とする機能を追加することさえできる.
    これにより,開発者と利用者が共になって新しいソフトウェアを創造することを通して,価値のあるプロダクトが生まれてきた.
	
	\begin{flushleft}
		■IT産業界の現状
	\end{flushleft}
    
         \begin{wrapfigure}{r}{7cm}
         		\begin{center}
		         \includegraphics[width=7cm]{figs/user_vendor_model.eps}
		         \caption{ユーザ企業とベンダ企業の構造}
		         \label{fig:user_vendor_model}
	         \end{center}
         \end{wrapfigure}
    
    しかしながら,産業界に目を向けると,我が国においては,コ・クリエイティブにソフトウェアを開発することよりも,
    IT技術を提供するベンダ企業と,自社のサービスのためにIT技術を利用するユーザ企業との間には対立構造が明確に存在し,
    両者のコンフリクトをマネジメントすることがソフトウェア開発チームに求められてきた.
    
    図\ref{fig:user_vendor_model}は,従来のソフトウェア開発におけるユーザ企業とベンダ企業との関係構造を模式化したものである.
    一番上に示した「エンドユーザ」とは,実際にソフトウェアを利用するユーザ(個人)である.エンドユーザは「ユーザ企業」に所属し,
    企業が提供するサービスを実現するために情報システムを利用する.近年は,B2C型でサービスを提供する企業が増えたことから,
    エンドユーザはユーザ企業の外部に存在し,Web等でユーザ企業が提供するサービスを利用する場合も見られる.
    
    このようなソフトウェア開発を行う場合,一般的にユーザ企業にある「情報システム部門」がシステム開発を主導することになる.
    情報システム部門は複数の「ベンダ企業」に対してRFP(Request For Proposal)を提示し,これを受けてベンダが作成した
    提案を精査し,ソフトウェア開発を発注するベンダ企業を選定する.この一連のプロセスはベンダ企業の「営業部門」が担当する.
    
    営業部門が契約を取り付けた後,ベンダ企業の「開発部門」が実際のソフトウェア開発プロジェクトを開始することになる.
    このとき,ベンダ企業内で必要なリソースが調達できない場合,ベンダ企業は社外の企業等に対して「アウトソース」を行う.
    いわゆる下請けの関係であり,近年は人件費の安い海外にアウトソースすることも多い.
    
	\begin{flushleft}
		■IT産業の構造変化
	\end{flushleft}

    前項で述べたとおり,既存のソフトウェア開発の産業構造では,ソフトウェアを実際に開発しているチームと,
    ソフトウェアを利用するエンドユーザとの間に,幾つもの障壁があることがわかる.
    このような構造でソフトウェア開発を進めている限り,利用者が本当に望むソフトウェア製品が開発される可能性が
    低くなるのは自明のことである.
    ましてや,マーケットとの対話を通してコ・クリエイティブに製品開発を進めることなど,既存の構造では不可能である.
    
    翻って世界に目を向けると,以上述べてきたユーザとベンダ企業が対立する構造によらない,新しいインターネット企業が登場してきている.例えば,
    GoogleやFacebookなどの有力な企業は,自らの顧客であるユーザとインタネットを通じて直接的にコミュニケーションをしながら,
    自社のプロダクトとしての情報サービスを提供している.加えて,App StoreやGoogle Playといったスマートフォン向けアプリの
    マーケットが登場しており,個人であっても直接ソフトウェアプロダクトをマーケットに投入することさえ容易になってきた.
    よって,今後は,従来の情報産業の枠組みに当てはまらない新しいタイプの企業が成長してくるものと予測する.
    
	\begin{flushleft}
		■次世代のソフトウェア開発者育成手法としてのPBL
	\end{flushleft}

    このような状況を踏まえると,今後は従来型の「ユーザ・ベンダ型モデル」は急速に存在感を失い,代わりに,
    ソフトウェアを開発するチームが直接的にマーケットとの対話を行い,より良いサービスを開発し,市場でマネタイズするという,
    「コ・クリエイティブ型のソフトウェア開発モデル」がより一般的になるとの確信に至る.
    
    そこで,本研究ではこのような「コ・クリエイティブ型ソフトウェア開発」に対応できる知識や技術を持った人材を育成するための
    PBL型の教材と教授法について研究開発を行うことを目的とする.
    
    従来のPBLは
    図\ref{fig:user_vendor_model}におけるベンダー企業の技術者育成を主眼とするものがほとんどである.
    これでは,産業構造の変化を踏まえた次世代の開発者を育成する内容として不十分である.
    特に,マーケットとのコ・クリエイティブな対話のプロセス,及び,
    迅速にソフトウェアを開発するチームとしてのアジャイル性を獲得する方法などについて,深く学べる内容にする必要がある.
    
    以上の背景を踏まえ,次世代のソフトウェア開発者を育成するための
    「コ・クリエイティブなソフトウェア開発者を育成するPBL型教育」の手法を確立し,必要な教材やWebサービスとともにパッケージ化し,
    様々な教育機関における教育に提供できる成果を得ることを本研究の目的とする.
    
    
	\vspace{1cm}
	\begin{thebibliography}{99}
		\bibitem{wired} 顧客とのco-creationプラットフォーム-ベストプラクティ, 2012/10/24参照 \\
                        \tt{http://wired.jp/2011/09/29/}
	\end{thebibliography}
%end  研究目的 ====================
}

%====================================
%form: kiban_ab_form_03-05.tex ; user: kiban_ab_03-05_plan.tex
%========== S-1-7 基盤研究(A,B)(一般) =========
%===== p. 03-05 研究計画・方法 =============
\section{研究計画・方法}
%watermark: w08_plan_AB
\newcommand{\研究計画と方法概要}{%
%begin  研究計画と方法概要===================
	研究全体はPDCAサイクル.
	
	Doの場所として,AIITのPBLと授業,SFC
	
	予想される成果は・・・
	電子書籍による教科書
	
	クラウドサービスを利用した,学習支援サービス
    最先端のAgile型ソフトウェア開発(Scrum)や,リーンスタートアップ等の最新の製品開発プロセス
    に基づき
    
%end  研究計画と方法概要 ====================
}

\newcommand{\研究計画}{%
%begin  研究計画===================
	
         \begin{wrapfigure}[11]{r}{7cm}
         	\begin{center}
		         \includegraphics[width=7cm]{figs/studio.eps}
		         \caption{教材制作スタジオ}
		         \label{fig:studio}
	         \end{center}
         \end{wrapfigure}
	
	\begin{flushleft}
		■研究全体の目標
	\end{flushleft}
	
	本研究は平成25年度から3カ年で実施し,全体を大きく次の目標に分割して取り組む.

	\begin{enumerate}
		\item \label{enum:startup} PBL教材制作環境
	    \item \label{enum:student} 学生向け事前学習教材
	    \item 教員向け指導手引書
	    \item 学習者支援用情報システム
	    \item 教育効果測定用キット
	    \item 成果発表
	\end{enumerate}
	
	このうち,\ref{enum:startup}から\ref{enum:student}までを平成25年度に実施し,残りを平成26年度以降に実施する.
	
	% 半年ごとにサイクルを回す.リーンスタートアップを実施する.

	\begin{flushleft}
		■平成25年度の計画
	\end{flushleft}
    

	\underline{PBL教材制作環境}とは,本PBLで使用する教材制作のためのスタジオ環境である.
	本研究で作成する教材は,音声や動画を用いた電子書籍とする.
	そこで,電子書籍教材制作に必要な映像・音響機器,及び,
	編集するためのコンピュータなどを購入して,本研究者の研究室に設置する.
	特に,作成する教材の映像・音響を収録のための機器は一定のクォリティ以上のものを選定する.
	図\ref{fig:studio}に示す通り、マイク、ビデオカメラ、ミキサ、スイッチャ、ライト等を配置し,
	教材制作のために使用する。
	
	% アナウンスブース,	テレプロンプター,	ビデオカメラ

	\underline{学生用事前学習教材}とは,PBLに入る前に事前に学習するための教材である.
	
	PBLでは,学生に事前の学習をするための教材が必要で,本研究者が実施した過去の例では
	いきなりソフトウェア開発プロジェクトを始めてもうまくいかない場合が多かった.
	そこで,本研究では学生に事前に準備のための学習をするための教材を	作成して提供する.
	
	コ・クリエイティブなソフトウェア開発者を育成するためのPBLの事前学習教材に含む内容は次のとおりとし,
	それぞれにInstructional Designを実施して学習者にとって理解しやすいように,ビジュアル面にも
	配慮したコンテンツとする.
	
         \begin{wrapfigure}[10]{r}{7cm}
     		\vspace{-2.5cm}
         	\begin{center}
		         \includegraphics[width=7cm]{figs/scrum.eps}
		         \caption{Scrumの全体像(吉羽氏資料より)}
		         \label{fig:scrum}
	         \end{center}
         \end{wrapfigure}

	\begin{enumerate}
	  \item \label{enum:scrum} Agile型開発プロセス「Scrum」の方法論とツール
	  \item \label{enum:lean} リーン・スタートアップの概念と実施法
	\end{enumerate}
	

	\ref{enum:scrum}.のScrumとは,他のウォータフォールモデルや,
	RUP(Rational Unified Process)などの方式と比較して,軽量な
	ソフトウェア開発のための方法論であり,近年注目されている.
	
	Scrumの全体像は図\ref{fig:scrum}でほぼ網羅されており,他の方式よりもシンプルであるため
	学習すべき知識の総量も少なくなる.しかしながら,実際には,単に知識として学ぶのではなく
	プロジェクトでScrumを実施できるようになるには相当の訓練が必要である.
	
	そこで,本研究で開発する教育法では
	知識項目を事前学習で学び,その後に続くPBLで実際にScrumをやってみることにより,Scrumで
	ソフトウェア開発を行うためのエッセンスを体得できるように工夫する.
	
	この教材に含む内容は,Scrumの全体概要,
	役割分担(Scrum MasterやProduct Owner,Team Memberなど),
	成果物(プロダクトバックログ,スプリントバックログ,バーンダウンチャートなど),
	プロセス(スプリント計画会議,デイリースクラム,振り返りなど)についてである.
	必要な学習時間は6~8時間を想定する.
	
	\ref{enum:lean}.のリーン・スタートアップとは,


	\begin{flushleft}
		■平成26年度以降の計画
	\end{flushleft}
	
	\underline{教員用指導手引書}
	
	\underline{学習者支援用情報システム}
	
	\underline{教育効果測定用キット}
	
	\underline{成果発表}
	
	これらの知見を学会等で発表し,本学における研究成果として社会に還元する
	
	各国語への翻訳も検討する.  
	
	関連書籍の購入
	PCの購入
	アルバイトの雇用
	教材収録用機材(カメラ・マイク・ミキサー)
	教材アンケート謝礼
	各国語への翻訳
	教材作成外注費
	
	教材作成スタジオ(マルチカメラ,タイムコード,足踏みスイッチ)
	リーンな教材開発


マーケットとの対話を通して、マーケットでマネタイズできる新しいサービスを、情報技術を活用して構築できる
プロジェクトを成功に導くための方法論
ソフトウェアの開発環境やコラボレーション・ツールを華麗に使いこなし、チームでがっちりスクラムを組み、ゴールを目指す

なぜ協創的ソフトウェア開発か
顧客とともに、顧客のためになるソフトウェアシステムを、顧客とともに開発するため
協創的なソフトウェア開発者とは
マーケットの動向がわかる
プロダクトのプロデュースができる
「迅速なチーム」でソフトウェア開発ができる
これらができる人材こそ求められている
この授業の目標
新しいソフトウェア開発方法論
従来型のソフトウェア開発プロセスを学ぶのではなく、マーケットとの対話を通してよりよいサービスを提供できるようになるためのベストプラクティス
協創型ソフトウェア開発を学ぶ
方法論と道具を学ぶ
実際に、方法論と道具を使ってみる
成果物をマーケットに投入してみる
従来型のソフトウェア開発
ユーザとベンダの構造
情報システムの開発組織体制
ベンダ企業が行う「Vモデル」
従来型のまとめ
ユーザ企業とベンダ企業で役割分担
ベンダ企業のマーケットはユーザ企業
実際にシステムを使う利用者は「エンドユーザ」と呼ばれる
上流工程の失敗がすべての下流工程に影響を与えるVモデルを利用

ソフトウェア開発をとりまく環境の変化
新しいプレイヤの登場
次の企業はユーザ企業・ベンダ企業のどちらだろうか?
Google
Facebook
Apple
GREE
楽天


アプリ・マーケットの登場
スマートフォンのアプリケーションを販売するマーケットの登場
App Store
Google Play
個人であってもマーケットにプロダクトを投入し、対価を得ることができる
クラウドインフラの整備
インフラに大きなコストが必要だった
インターネットの常時接続回線を確保
サーバのハードウェアを設置する
サーバの保守・管理を行う
クラウド技術の発展
非常に安価に、場合によっては無料で仮想のサーバをインターネット上に設置できる
Amazon EC2
Heroku
Sakura VPSなど
オープンソース型開発の世界
オープンソースの開発体制は、今も昔も、明らかに、「ユーザ・ベンダ型モデル」に当てはまらない
自らが欲する機能(サービス)を手に入れるため、自らがコミュニティに参加してソフトウェアを開発する
従来は、コミュニティをホストする環境を用意するためのコストが高かったが、近年はクラウドによりそれが低くなっている
アジャイル型ソフトウェア開発
アジャイルソフトウェア開発宣言
2001年に作成されたアジャイルソフトウェア開発手法のエッセンスをまとめた文書
従来のVモデル(ウォータフォール開発モデル)に対するアンチテーゼ
10年以上前に作成されたもので、今の視点から見れば若干古臭くはあるものの、その趣旨は大いに参考になる。
「顧客」という言葉を「マーケットにいるお客さま」に置き換えれば、ほぼ、整合性はとれる。

アジャイルソフトウェア開発の価値

プロセスやツールより人と人同士の相互作用を重視する。
包括的なドキュメントより動作するソフトウェアを重視する。
契約上の交渉よりも顧客との協調を重視する。
計画に従うことよりも変化に対応することを重視する。


アジャイルソフトウェア開発の原則1
最も重要なことは顧客を満足させること。早く、そして継続的に、価値のあるソフトウェアをリリースする。
開発の終盤においても要求の変更を受け入れる。アジャイルプロセスは顧客の競争力を優位にするための道具である。
数週間、数ヶ月の単位で頻繁に実用的なソフトウェアをリリースする。タイムスケールは短い方がよい。
プロジェクトの間中、毎日、顧客と開発者は一緒に働くべきである。
やる気のある人を中心にプロジェクトを構築する。環境と必要なサポートを与え、彼らが仕事を成し遂げると信じること。
アジャイルソフトウェア開発の原則2
開発チーム内で情報伝達を行う効果的で有効な方法は、Face to Faceによる会話である。
進捗を測るには、動くソフトウェアが一番である。
アジャイルプロセスは、継続的な開発を促進する。スポンサー、開発者そしてユーザは一定のペースを保つようになる。
優れた技術と良い設計に絶えず目を配ることで、機敏になる。
単純性--最大限に仕事を行わないことは極めて重要である。
最良のアーキテクチャは自己最適化されたチームから現れる。
定期的な間隔で、チームにもっとも効果的な方法を反映することで、調律・調整に従うようになる。

ソフトウェア開発をとりまく環境の変化のまとめ

近年は、自らサービスを開発してマーケットに投入し、マネタイズするタイプのプレイヤが成長している
アプリケーションのマーケットが確立され、容易にプロダクトを販売できるようになった
クラウドインフラが整備されてきた
アジャイル開発の価値が浸透し、開発スタイルが変化しつつある
授業で学ぶ方法論と道具
この授業で学ぶ方法論とツール
ソフトウェア開発方法論
Agile型ソフトウェア開発の手法である「Scrum」について学び、実践できるようになる
ソフトウェア開発環境
コラボレーティブにチーム開発をするための環境である「GitHub」について学ぶ
クラスプラットフォームの開発環境(enchant.jsを予定)を学ぶ
Scrumの全体像
GitHubとは
オープンソース型開発のために編み出された知恵や工夫を具現化したコラボレーションのためのツール
授業ではGitHub用の専用アプリケーションを用いる
クロスプラットフォーム開発環境
現時点で主要なプラットフォーム
Windows/Mac/Linux
iPhone/iPad
Android Phone/Tablet
その他
クロスプラットフォーム開発の必要性
多くのプラットフォームで動作するソフトウェアを一度に作成することができる
産学連携型の教育
社会人の参加
この授業では、社会人がScrum MasterやProduct Owner、あるいは、メンターやレビューワ、コーチの形で参加してもらいます。
社会人が参加することの意義
社会人がもつ実務的な経験を学ぶことができます。
社会人にとっても学生の皆さんと協創的にものづくりを行うことは刺激になります。
授業の進め方
講義とPBL型演習
スキル取得(最初の3回)
ガイダンス
開発環境とコラボレーション・ツール
Scrum型開発方法論
PBL型演習の実施
PBL = Project Based Learning
社会人の参加
成績評価
GitHubの課題アクション
Scrum会議への参加状況
作り上げたソフトウェアサービスそのもの
マーケットからの評価
最終レポート
成果発表会・親睦会
最終成果発表会
学外からレビューワーを招き、講評して頂く機会を設けますので、参加すること
2/22(金) or 2/23(土)
懇親会
チームによるプロジェクトではメンバーの団結力が重要である
第3週の授業後に開催予定
社会人も参加予定


    
    また、象の卵の殻の仕組みが解明されれば、
	\begin{itemize}
		\item 象の生態の解明、恐竜の卵の構造の理解(生物学)、
		\item 殻の化学生成反応の解明(化学)、
		\item 殻の原子レベルでの構造とC$_{60}$やナノクラスターとの関連の研究(物理)、
		\item 人工的に象の殻を作り、車の車体などに応用できる(工学)
	\end{itemize}
	など、科学、社会への影響は計り知れない。

         \begin{wraptable}{r}{0.6\linewidth}
         		\caption{各種動物の、足一本にかかる平均加重}
		\label{tab:load}
         		\begin{tabular}{lrrr}
			\hline
			動物 & 体重 & 足の本数 & 加重(kg/足)\\
			\hline
			ジョロウグモ	& 20mg	& 8	& 2.5mg \\
			象 			& 5t & 4	& 1.3t \\
			人間 			& 60kg	& 2	& 30kg\\
			フラミンゴ	& 10kg	& 1	& 10kg\\
			キングコブラ	& 7kg	& 0	& $\infty$\\
			\hline
		\end{tabular}
         \end{wraptable}

         紀元前に、アルキメデス(\('A\rho\chi i\mu\acute{\eta}\delta\eta\mbox{\c{c}}\))は
	象の卵の形を円筒座標表示で
         \[r(z) = 0.5\sqrt{1-(e^z-2)^2}\]
         で近似し、その体積を求めようとしたが、当時はまだ
         \begin{equation}
	         V  = \pi \int_0^{\ln 3} r^2(z) dz\\
         \end{equation}
         の計算が難しくあきらめていた。
         しかしある日、好物の温泉卵を作ろうとして鶏の卵を持って入浴している最中に、
         風呂からあふれるお湯を見て、象の卵の体積を測定する方法を思いついたと言われる。
 
	さて、象の卵の殻の強度については、すでに19世紀初めにロシアのキーファ・モキエーイチが
	考察していると、ゴーゴリが紹介している
	\cite{gogori}。
	しかし、この斬新で自由な発想にもとづく科学的考察に対し、
	トルストイは果敢にも、
	そういう考察がいかに論理的であろうとそれ自体間違っていて無駄である、
	と厳しく批判している
	\cite{torusutoi}。
	これは、既成概念にとらわれた、科学に対する挑戦ともとれるが、
	まだ進化論が現代の米国のように広く信じられていなかった
	帝政ロシアの時代にあっては、

	例えば逢坂北部のある終点駅の駅前では、
	毎年年末になると図\ref{fig:egg_R}, \ref{fig:egg_L}に示すように
	象の卵の像のまわりを電飾するしきたりが残っている。
	(少し寄り目にし、右目で左の図、左目で右の図を見てください。
	なお、このように図や表を横に並べる方が、{\tt wrapfigure}を用いるより位置の調整が楽です。)
        \begin{figure}[h]
         	\begin{minipage}[t]{0.49\linewidth}
			\includegraphics[width=\linewidth]{figs/egg_R.eps}
			\caption{右目用}
			\label{fig:egg_R}
		\end{minipage}
		\hspace{0.01\linewidth}
		\begin{minipage}[t]{0.49\linewidth}
			\includegraphics[width=\linewidth]{figs/egg_L.eps}
			\caption{左目用}
			\label{fig:egg_L}
		\end{minipage}
         \end{figure}

	また、寺村輝夫の研究\cite{teramura}によれば、昔、
	王子の誕生を祝って国民全員に卵焼きを提供すべく、
	軍隊を動員して象の卵を探させた王がいた。
	このときは孵化直後の子象は見つかったが、それが入っていた殻の発見には至っていない。
	人の家の裏庭の犬小屋を衛星写真で調べることさえもできなかった時代とあっては、
	この失敗も無理からぬことである。
	
	しかし今や、進化論は確立し、遺伝子の解析による派生の系統解析や
	犯人の特定ができる時代である。
	また、土を掘り返すことを基本としていた考古学でも、
	宇宙からナスカの近くに新たな地上絵を発見する時代である。
	このように、
	現代の科学技術を駆使すれば、マクロな広範囲に渡る精細な探索と、
	ミクロな遺伝子からの解析は可能であり、
	象の卵を世界に先駆けて発見することは、科学技術立国としての日本に課せられた使命でもある
	と言っても過言ではない。

	初年度は、まず世界の動物園を巡り、
	研究業績 \KLcite{pub:theoegg}に可能性が示されたように
	象舍に卵が隠されていないか、探す。

	2年目はアフリカに行き、空と地上から象の卵を探す。
	アフリカ象は気性が荒いが、サバンナの方がジャングルよりも見通しが効くので、
	インドよりもアフリカを先に探索する。

	3年目は、インドとタイに行き、ジャングルに隠されている卵を探す。
	ジャングルの場合は空からは探しにくいが、象使いも多く、象の背中に乗って
	象の視点から探索することができる。
	さらに、気だての優しいインド象ならば
	卵の在処を教えてくれる可能性もある。
	
	\vspace{1cm}
	\begin{thebibliography}{99}
		\bibitem{gogori} ゴーゴリ、「死せる魂」(1841).
		\bibitem{torusutoi} トルストイ、「人生論」(1886).
		\bibitem{teramura} 寺村輝夫、「ぼくは王様 - ぞうのたまごのたまごやき」.
	\end{thebibliography}
	
	クラウド型LMSを構築する.
%end  研究計画 ====================
}

%form: kiban_ab_form_06.tex ; user: kiban_ab_06_preparation_final_year.tex
%========== S-1-7 基盤研究(A,B)(一般) =========
%===== p. 06 準備状況等、最終年度の応募 =============
\section{準備状況等、最終年度の応募}
\subsection{準備状況等}
\newcommand{\準備状況等}{%
%begin  準備状況等 ===================
	\underline{本研究を実施するための研究施設}としては,
	産業技術大学院大学(AIIT)では2006年度より情報システムのアーキテクトを育成するための
	PBLを実施しており(研究業績の\KLcite{pub:tozawa-pbl-2009}),本研究はこの一環として施設等を利用できる.
	また,このPBLにおいて,本研究者らはソフトウェア開発方法論を教育する目的で,
	反復型開発プロセスであるRUP(Rational Unified Process)や,XP(eXtreme Programming),チケット駆動開発などを
	指導した実績を有し,ここから得られた知見も活用する.
	特に,2009年度以降は,ベトナム国家大学の学生と共にグローバルPBLを展開し,
	海外の技術者との共同プロジェクトを実施し,その成果を発表している
	\KLcite{pub:kizaki-global-2011a}\KLcite{pub:kizaki-global-2011b}\KLcite{pub:kizaki-global-2011c}%
	\KLcite{pub:chubachi-global-2010}%
	\KLcite{pub:ohrui-global-2009}\KLcite{pub:tozawa-global-2009}.
	
	加えて,慶應義塾で開講している「協創型ソフトウェア開発」の授業を2011年度から担当し,今年度からは
	アジャイル型ソフトウェア開発手法であるScrumを全面的に導入し,コ・クリエイティブなソフトウェア開発者教育を始めたところである.

	\underline{本研究の研究成果を発信}するためには,AIITにおけるPBL全体を支援する情報インフラストラクチャに関する研究の成果
	\KLcite{pub:chubachi-ipbl-2012}\KLcite{pub:chubachi-ipbl-2011}%
	\KLcite{pub:chubachi-ipbl-2009a}\KLcite{pub:chubachi-ipbl-2009b}
	を活用する.
%end  準備状況等 ====================
}

\subsection{研究計画最終年度の応募}
\newcommand{\研究計画最終年度の応募の研究種目名}{%
%begin  研究種目名 ===================
         % 基盤研究A
%end  研究種目名 ====================
}

\newcommand{\研究計画最終年度の応募の審査区分}{%
%begin  審査区分 ===================
	%123
%end  審査区分 ====================
}

\newcommand{\研究計画最終年度の応募の課題番号}{%
%begin  課題番号 ===================
    %12345678	%半角(英数字)
%end  課題番号 ====================
}

\newcommand{\研究計画最終年度の応募の研究課題名}{%
%begin  研究課題名 ===================
	%シロナガスクジラの卵の殻はなぜ見つからないのか
%end  研究課題名 ====================
}

\newcommand{\研究計画最終年度の応募の研究期間初年度}{%
%begin  研究期間初年度 ===================
	%15
%end  研究期間初年度 ====================
}

\newcommand{\研究計画最終年度の応募の計画と成果}{%
%begin  特別推進研究又は基盤研究による研究計画及び研究成果 ===================
	%研究課題の通り、シロナガスクジラの卵は見つけられなかった。
%end  特別推進研究又は基盤研究による研究計画及び研究成果 ====================
}

\newcommand{\研究計画最終年度の応募の理由}{%
%begin  研究計画最終年度前年度の応募をする理由 ===================
	%さっさと次の研究に移りたいので。
%end  研究計画最終年度前年度の応募をする理由 ====================
}

%====== end of page =====================================
%form: kiban_ab_form_07-09.tex ; user: kiban_ab_07-09_publications.tex
%========== S-1-7 基盤研究(A,B)(一般) =========
%===== p. 07-09 研究業績 =============
\section{研究業績}
%watermark: w14_pub_AB
% 2012-09-01 Taku
\newcommand{\年と名前と研究業績}{%
%begin  研究業績 ===================
		% 2013年度から始まった2カラムのtabularです。
%ーーーーーーーーーーーーーーーーーーーーーーーーーーーーーーーーーーーーーーーーーーーー
%		\KLcite{pub:theoegg} のようにして業績番号を文中に入れられます。
%ーーーーーーーーーーーーーーーーーーーーーーーーーーーーーーーーーーーーーーーーーーーー
	2012 {\small 以降} \\
		中鉢欣秀
		& \KLbibitem \label{pub:chubachi-ipbl-2012} \me, 小山裕司: AIITにおけるプロジェクト型学修(PBL)のためのBacklogシステムの導入, 情報処理学会 第19回IOT・第39回EVA合同研究発表会, 島根県松江市, 2012-09-27\\ 
	\hline%----------------------------------------------
	
	2011 \\
		中鉢欣秀
		& \KLbibitem \label{pub:chubachi-ipbl-2011}\me, 小山 裕司: PBLを支援するコラボレーティブツールに関する考察, 産業技術大学院大学紀要, No.5,pp.100-108, 2011 (査読有)\\
		% & \KLbibitem 小山 裕司, \me: 外部アカウント認証を使った本人確認付き利用者認証の仕組み, 産業技術大学院大学紀要, No.5,pp.75-80, 2011(査読有) \\
		& \KLbibitem \me: 目的/手段展開に基づくソフトウェアアーキテクチャの仕様化, 要求工学WGワークショップ, 情報処理学会, 礼文島, 2011-06-24 \\
		& \KLbibitem \label{pub:kizaki-global-2011a} 木崎 悟, 成田 亮, 丸山 英通, \me: グローバルなソフトウェア開発におけるマネジメント手法, 情報処理学会 第172回ソフトウェア工学研究会, 早稲田大学, 2011-05-17 \\
		& \KLbibitem \label{pub:kizaki-global-2011b} 木崎 悟, 成田 亮, 丸山 英通, 土屋 陽介, 成田 雅彦, \me: 国際PBLにおける的確な仕様の伝達とチケット駆動による開発作業の効率化, ソフトウェアエンジニアリングシンポジウム2011, 東京女子大学, 2011-09. \\
		& \KLbibitem \label{pub:kizaki-global-2011c} 木崎 悟, 丸山 英通, 土屋 陽介, \me: ソフトウェア開発PBLへのチケット駆動開発の適用による共同作業の改善, プロジェクトマネジメント学会 2011年度秋季研究発表大会, 産業技術大学院大学, 2011-09. \\
		& \KLbibitem 小山 裕司, \me, 土屋 陽介: ソーシャルメディアを活用したコネクション構築支援, 情報処理学会研究報告. コンピュータと教育研究会報告, 一般社団法人情報処理学会, Vol.2011, No.3, pp.1-6, 2011-12-10. \\
		& \KLbibitem 土屋 陽介, 小山 裕司, \me: 授業配信システムの設計と開発, 情報処理学会研究報告. コンピュータと教育研究会報告, 一般社団法人情報処理学会, Vol.2011, No.2, pp. 1-7, 2011-12-10. \\
		& \KLbibitem \me, 小山 裕司, 石島 辰太郎: 産業技術大学院大学のICT環境の運用と課題, 研究報告インターネットと運用技術(IOT), 一般社団法人情報処理学会, Vol.2012-IOT-16, No.11, pp.1-4, 2012-03-08 \\

	\hline%----------------------------------------------
	2010 \\
		中鉢欣秀
		&  \KLbibitem \label{pub:chubachi-global-2010} \me, 成田 雅彦, 戸沢 義夫: 加藤由花, 戸沢義夫: ベトナム国家大学とのグローバル PBL から得た知見, 産業技術大学院大学紀要, pp.1-4, 2010 (査読有) \\
		&  \KLbibitem S.~Ishijima, H.~Koyama, \meen, F.~Harashima: ICT-based Learning System of AIIT for the professional education in Japan, 9th International Conference on Information Technology Based Higher Education and Training (ITHET 2010), 2010-04-29 \\
		&  \KLbibitem R.~Nishino, M.~Kojima, O.~Oka, T.~Okino, T.~Sugita, Y.~Tsuchiya, H.~Koyama, Y.~Tozawa, \meen: Experience Gained through International PBL in Software Development, 1st Asia-Pacific Joint PBL Conference 2010, 2010-10-23 \\
		&  \KLbibitem \meen, Y.~Kato, Y.~Tozawa: Web-based groupware supporting PBL effectively, 1st Asia-Pacific Joint PBL Conference 2010, 2010-10-24 \\
		&  \KLbibitem \me: ワークショップ実行委員長業務のSBVA法による要求分析, 要求工学ワーキンググループ ワークショップ, 情報処理学会, 礼文島, 2010-06-17 \\
		&  \KLbibitem 木崎 悟, 成田 亮, 丸山 英通, \me, 長尾 雄行: GTD初心者のタスク管理を支援するタスクコンシェルジュの開発, 第9回情報科学技術フォーラム, 福岡県福岡市, 2010-08-20 \\
		&  \KLbibitem \me, 小山 裕司, 石島 辰太郎: ICTを基盤とした高度専門職教育, 情報教育シンポジウム論文集, 情報処理学会, 情報処理学会シンポジウムシリーズ IPSJ Symposium Series Vol.2010, No.6, pp.133-138, 群馬県渋川市, 2010-08-19 \\
		&  \KLbibitem \me: 遠隔会議システムを用いた国際PBLから得た知見, 日本e-Learning学会 2010年度学術講演会論文誌, 東京都千代田区, 2010-11-14 \\
		&  \KLbibitem 木崎 悟, 成田 亮, 丸山 英通, 土屋陽介, \me: タスク管理を支援するタスクコンシェルジュの開発, 電子情報通信学会総合大会ポスターセッション, 東京都市大学, 2011-03-16. \\

	\hline%----------------------------------------------

	2009 \\
		中鉢欣秀
		 &  \KLbibitem \label{pub:chubachi-ipbl-2009a} \me, 土屋 陽介, 長尾 雄行, 加藤 由花, 酒森 潔, 戸沢 義夫: グループウェア導入によるPBLの見える化, 日本e-Learning学会論文誌, Vol.9, pp.129-135, 2009-05(査読有) \\
		 &  \KLbibitem \label{pub:chubachi-ipbl-2009b} \me, 加藤由花, 戸沢義夫: PBL用情報インフラストラクチャの構築と運用, 産業技術大学院大学紀要, pp.109-116, 2009 (査読有) \\
		 &  \KLbibitem \label{pub:tozawa-pbl-2009} Y.~Tozawa, Y.~Kato, \meen: Efforts to ensure the quality of PBL education in the graduate school of Information Technology, Proceedings of the 2nd International Research Symposium on PBL, 3-4 December 2009, Melbourne, Australia, pp.1-9 \\
		 &  \KLbibitem \label{pub:ohrui-global-2009} 大類 優子,成田 雅彦,\me,土屋 陽介,戸沢 義夫: Global PBL Feasibility Studyの実践検証, 情報科学技術フォーラム講演論文集, FIT(電子情報通信学会・情報処理学会)推進委員会, 2009-08-20, Vol.8, No.4, pp. 515-516 \\
		 &  \KLbibitem \me: 要求記述演習によるロジカルシンキング教育の評価, 要求工学ワーキンググループ ワークショップ, 情報処理学会, 銚子, 2009-05-29 \\
		 &  \KLbibitem \label{pub:tozawa-global-2009} 戸沢 義夫, 成田 雅彦, \me, 土屋 陽介: Global PBL Feasibility Studyの実践と得られた知見, 情報処理学会 情報教育シンポジウム論文集, pp.167-174,2009-08-20 \\
		 &  \KLbibitem \me: 要求分析モデリング支援システムの開発~SBVAエディタ~, 要求工学ワーキンググループ ワークショップ, 情報処理学会, 天橋立, 2009-10-22 \\
		 % &  \KLbibitem \me: 要求工学セッションの紹介~WW2010~, ウィンターワークショップ2010イン・倉敷論文集,情報処理学会,情報処理学会シンポジウムシリーズ IPSJ Symposium Series Vol.2010,No.3, pp.31-32, 2010-01-21 \\

	\hline%----------------------------------------------

	2008 \\
		中鉢欣秀
		&  \KLbibitem 長尾 雄行, 土屋 陽介, 森本 祥一, \me: JavaScriptと非同期HTTPリクエストによる共同作業支援ミドルウェアの構築, 産業技術大学院大学紀要, Vol.2, pp.165-174, 2008 \\
		&  \KLbibitem 森本 祥一, \me: シナリオの図解化による業務フロー分析, 産業技術大学院大学紀要, Vol.2, pp.193-208, 2008 \\
		&  \KLbibitem \me, 専門職大学院におけるモデリング教育とSBVA法, 要求工学ワーキンググループ ワークショップ, 情報処理学会, 奄美大島, 2008-05-15 \\
		&  \KLbibitem 長尾 雄行, 土屋 陽介, 森本 祥一, \me: JavaScriptと非同期HTTPリクエストによる共同作業支援ミドウェアの構築, 情報処理学会論文誌:プログラミング, Vol.1, No. 1, pp.63-64, 2008-06 \\
		&  \KLbibitem \me, システム開発における仮説検証型の要求分析プロセス, 要求工学ワーキンググループ ワークショップ, 情報処理学会, 雲仙, 2008-10-23 \\
		&  \KLbibitem \me, 土屋 陽介, 長尾 雄行, 加藤 由花, 酒森 潔, 戸沢 義夫, PBLを見える化する協調作業支援環境の構築, 日本e-Learning学会2008年秋季学術講演会論文集, pp.72-79, 京都, 2008-11 ※優秀賞受賞 \\
		&  \KLbibitem \me: 要求分析者育成のためのコミュニケーション能力教育, ウィンターワークショップ2009・イン・宮崎論文集, 情報処理学会, Vol.2009, No.3, pp.45-46, 宮崎, 2009-01-23 \\
		% &  \KLbibitem 橋山 牧人,中鉢 欣秀, 大岩 元: 携帯ゲームアプリケーション開発を支援するオブジェクト指向を用いたフレームワークの開発, 情報処理学会第71会全国大会, pp.4-733-734,草津(滋賀), 2009年3月 (NII)
%end  研究業績  ====================
}

\newcommand{\連携研究者の研究業績}{%
%begin  連携研究者の研究業績 ===================
		% 2カラムのtabularです。
%ーーーーーーーーーーーーーーーーーーーーーーーーーーーーーーーーーーーーーーーーーーーー
%		\KLciteB{pub:theoegg} のようにして業績番号を文中に入れられます。
%ーーーーーーーーーーーーーーーーーーーーーーーーーーーーーーーーーーーーーーーーーーーー
%end  連携研究者の研究業績  ====================
}

%===========================================================
 %=======================================================
%form: kiban_ab_form_10.tex ; user: kiban_ab_10_past_funds.tex
%========== S-1-7 基盤研究(A,B)(一般) =========
%===== p. 10 これまでに受けた研究費とその成果等 =============
\section{これまでに受けた研究費とその成果等}
\newcommand{\これまでに受けた研究費とその成果等}{%
%begin  これまでに受けた研究費とその成果等 ===================
	\begin{itemize}
		\item 若手研究(B) ,2008~2009年度,
		    「情報システムアーキテクト育成のための遠隔教育システム」,研究代表者,3,900千円\\
		    本研究では社会人教育における利用を想定したモデリング遠隔教育支援シス
            テムを研究開発した.これを用いて,特にユーザ企業の社会人を対象としたモデリング
            教育支援環境を構築し,その有用性を確かめることができた.
    \end{itemize}
%end  これまでに受けた研究費とその成果等 ====================
}

%====== end of page =====================================
%form: kiban_ab_form_11.tex ; user: kiban_ab_11_relation.tex
%========== S-1-7 基盤研究(A,B)(一般) =========
%===== p. 11 研究計画と研究進捗評価を受けた研究課題の関連性 =============
\section{研究計画と研究進捗評価を受けた研究課題の関連性}
\newcommand{\研究計画と研究進捗評価を受けた研究課題の関連性}{%
%begin  研究計画と研究進捗評価を受けた研究課題の関連性 ===================
	特になし.
%end  研究計画と研究進捗評価を受けた研究課題の関連性 ====================
}

%====== end of page =====================================
%form: kiban_ab_form_12.tex ; user: kiban_ab_12_human_rights_etc.tex
%========== S-1-7 基盤研究(A,B)(一般) =========
%===== p. 12 人権、法令、研究経費の妥当性など =============
\section{人権、法令、研究経費の妥当性など}
\newcommand{\人権の保護及び法令等の遵守への対応}{%
%begin  人権の保護及び法令等の遵守への対応 ===================
	特になし.
%end  人権の保護及び法令等の遵守への対応 ====================
}

\subsection{研究経費の妥当性・必要性}
\newcommand{\研究経費の妥当性と必要性}{%
%begin  研究経費の妥当性・必要性 ===================
	「研究計画・方法」欄で述べた研究規模、研究体制等を踏まえると、
	次頁以降に記入する研究費は妥当、かつ必要であり、
	積算根拠も妥当である。
%end  研究経費の妥当性・必要性 ====================
}

%====== end of page =====================================
%form: kiban_ab_form_13.tex ; user: kiban_ab_13_materials.tex
%========== S-1-7 基盤研究(A,B)(一般) =========
%===== p. 13 設備備品費、消耗品費の明細 =============
\section{設備備品費、消耗品費の明細}
%				\KLJFY{\1年目J}
\newcommand{\設備備品費1年目}{%
%begin  設備備品費1年目 ===================
	\KLItemNumUnitCostLocation{卓上マイク}{2}{35}{産技大}
	\KLItemNumUnitCostLocation{音響ミキサー}{1}{60}{産技大}
	\KLItemNumUnitCostLocation{音響用ケーブル一式}{1}{100}{産技大}
	\KLItemNumUnitCostLocation{ビデオカメラ}{2}{115}{産技大}
	\KLItemNumUnitCostLocation{ビデオカメラスタンド}{2}{40}{産技大}
	\KLItemNumUnitCostLocation{映像用スイッチャ}{1}{120}{産技大}
	\KLItemNumUnitCostLocation{映像用モニタ}{2}{45}{産技大}
	\KLItemNumUnitCostLocation{スタンド式ライト}{2}{60}{産技大}
	
%end  設備備品費1年目 ====================
}

\newcommand{\消耗品費1年目}{%
%begin  消耗品費1年目 ===================
	% 2カラムのtabularです。
	\KLItemCost{タケコプター燃料}{56789}
	\KLItemCost{象の餌代}{10000}
	\KLItemCost{卵切断用鋸}{1000}
%end  消耗品費1年目 ====================
}

\newcommand{\設備備品費2年目}{%
%begin  設備備品費2年目 ===================
	\KLItemNumUnitCostLocation{タケコプター}{2}{123000}{ケニア大学}
	\KLItemNumUnitCostLocation{大型フライパン}{2}{20}{どこでもよい}
%end  設備備品費2年目 ====================
}

\newcommand{\消耗品費2年目}{%
%begin  消耗品費2年目 ===================
	\KLItemCost{タケコプター燃料}{80000}
	\KLItemCost{象の餌代}{20000}
	\KLItemCost{ハードディスク}{2000}
%end  消耗品費2年目 ====================
}

\newcommand{\設備備品費3年目}{%
%begin  設備備品費3年目 ===================
	\KLItemNumUnitCostLocation{タケコプター}{3}{123000}{ケニア大学}
	\KLItemNumUnitCostLocation{大型フライパン}{3}{20}{どこでもよい}
%end  設備備品費3年目 ====================
}

\newcommand{\消耗品費3年目}{%
%begin  消耗品費3年目 ===================
	\KLItemCost{象の餌代}{30000}
	\KLItemCost{ハードディスク}{3000}
%end  消耗品費3年目 ====================
}

\newcommand{\設備備品費4年目}{%
%begin  設備備品費4年目 ===================
	\KLItemNumUnitCostLocation{タケコプター}{4}{123000}{ケニア大学}
	\KLItemNumUnitCostLocation{大型フライパン}{4}{20}{どこでもよい}
%end  設備備品費4年目 ====================
}

\newcommand{\消耗品費4年目}{%
%begin  消耗品費4年目 ===================
	\KLItemCost{象の餌代}{40000}
	\KLItemCost{ハードディスク}{4000}
%end  消耗品費4年目 ====================
}

\newcommand{\設備備品費5年目}{%
%begin  設備備品費5年目 ===================
	\KLItemNumUnitCostLocation{タケコプター}{5}{123000}{ケニア大学}
	\KLItemNumUnitCostLocation{大型フライパン}{5}{20}{どこでもよい}
%end  設備備品費5年目 ====================
}

\newcommand{\消耗品費5年目}{%
%begin  消耗品費5年目 ===================
	\KLItemCost{象の餌代}{50000}
	\KLItemCost{ハードディスク}{5000}
%end  消耗品費5年目 ====================
}

%====== end of page =====================================
%form: kiban_ab_form_14.tex ; user: kiban_ab_14_travels.tex
%========== S-1-7 基盤研究(A,B)(一般) =========
%===== p. 14 旅費等の明細 =============
\section{旅費等の明細}
%	\KLNoMarginMinipage{66}{706}{566}{
%				\KLJFY{\1年目J}
\newcommand{\国内旅費1年目}{%
%begin  国内旅費1年目 ===================
		% 2カラムのtabularです。
		\KLItemCost{探検打合わせ}{150}
		\KLItemCost{象の調査}{120}
%end  国内旅費1年目 ====================
}

\newcommand{\外国旅費1年目}{%
%begin  外国旅費1年目 ===================
		% 2カラムのtabularです。
		\KLItemCost{卵収集}{1500}
		\KLItemCost{象の調査}{1200}
%end  外国旅費1年目 ====================
}

\newcommand{\謝金等1年目}{%
%begin  謝金等1年目 ===================
		% 2カラムのtabularです。
	 	\KLItemCost{アルバイト報酬(テープ起こし等)}{100}
		\KLItemCost{講師謝礼(Scrumコーチ等)}{300}
%end  謝金等1年目 ====================
}

\newcommand{\その他1年目}{%
%begin  その他1年目 ===================
		% 2カラムのtabularです。
	 	\KLItemCost{通信費}{800}
		\KLItemCost{卵運搬費}{4000}
		\KLItemCost{ジープ借料}{4100}
%end  その他1年目 ====================
}

\newcommand{\国内旅費2年目}{%
%begin  国内旅費2年目 ===================
		% 2カラムのtabularです。
		\KLItemCost{探検打合わせ}{250}
		\KLItemCost{象の調査}{220}
%end  国内旅費2年目 ====================
}

\newcommand{\外国旅費2年目}{%
%begin  外国旅費2年目 ===================
		% 2カラムのtabularです。
		\KLItemCost{卵収集}{2500}
		\KLItemCost{象の調査}{2200}
%end  外国旅費2年目 ====================
}

\newcommand{\謝金等2年目}{%
%begin  謝金等2年目 ===================
	 	\KLItemCost{パイロット報酬}{3000}
		\KLItemCost{ハンター賃金}{2000}
%end  謝金等2年目 ====================
}

\newcommand{\その他2年目}{%
%begin  その他2年目 ===================
	 	\KLItemCost{通信費}{800}
		\KLItemCost{卵運搬費}{4000}
		\KLItemCost{ジープ借料}{4200}
%end  その他2年目 ====================
}

\newcommand{\国内旅費3年目}{%
%begin  国内旅費3年目 ===================
		% 2カラムのtabularです。
		\KLItemCost{探検打合わせ}{350}
		\KLItemCost{象の調査}{320}
%end  国内旅費3年目 ====================
}

\newcommand{\外国旅費3年目}{%
%begin  外国旅費3年目 ===================
		% 2カラムのtabularです。
		\KLItemCost{卵収集}{3500}
		\KLItemCost{象の調査}{3200}
%end  外国旅費3年目 ====================
}

\newcommand{\謝金等3年目}{%
%begin  謝金等3年目 ===================
	 	\KLItemCost{パイロット報酬}{3000}
		\KLItemCost{ハンター賃金}{3000}
%end  謝金等3年目 ====================
}

\newcommand{\その他3年目}{%
%begin  その他3年目 ===================
	 	\KLItemCost{通信費}{800}
		\KLItemCost{卵運搬費}{4000}
		\KLItemCost{ジープ借料}{4300}
%end  その他3年目 ====================
}

\newcommand{\国内旅費4年目}{%
%begin  国内旅費4年目 ===================
		% 2カラムのtabularです。
		\KLItemCost{探検打合わせ}{450}
		\KLItemCost{象の調査}{420}
%end  国内旅費4年目 ====================
}

\newcommand{\外国旅費4年目}{%
%begin  外国旅費4年目 ===================
		% 2カラムのtabularです。
		\KLItemCost{卵収集}{4500}
		\KLItemCost{象の調査}{4200}
%end  外国旅費4年目 ====================
}

\newcommand{\謝金等4年目}{%
%begin  謝金等4年目 ===================
	 	\KLItemCost{パイロット報酬}{3000}
		\KLItemCost{ハンター賃金}{4000}
%end  謝金等4年目 ====================
}

\newcommand{\その他4年目}{%
%begin  その他4年目 ===================
	 	\KLItemCost{通信費}{800}
		\KLItemCost{卵保管費}{4400}
%end  その他4年目 ====================
}

\newcommand{\国内旅費5年目}{%
%begin  国内旅費5年目 ===================
		% 2カラムのtabularです。
		\KLItemCost{探検打合わせ}{550}
		\KLItemCost{象の調査}{520}
%end  国内旅費5年目 ====================
}

\newcommand{\外国旅費5年目}{%
%begin  外国旅費5年目 ===================
		% 2カラムのtabularです。
		\KLItemCost{卵収集}{5500}
		\KLItemCost{象の調査}{5200}
%end  外国旅費5年目 ====================
}

\newcommand{\謝金等5年目}{%
%begin  謝金等5年目 ===================
	 	\KLItemCost{パイロット報酬}{3000}
		\KLItemCost{ハンター賃金}{5000}
%end  謝金等5年目 ====================
}

\newcommand{\その他5年目}{%
%begin  その他5年目 ===================
	 	\KLItemCost{通信費}{800}
		\KLItemCost{卵保管費}{4500}
%end  その他5年目 ====================
}

%====== end of page =====================================
%form: kiban_ab_form_15.tex ; user: kiban_ab_15_other_applications.tex
%========== S-1-7 基盤研究(A,B)(一般) =========
%===== p. 15 研究費の応募・受入等の状況・エフォート =============
\section{研究費の応募・受入等の状況・エフォート}
\subsection{応募中の研究費}
\newcommand{\本人の研究経費}{%
%begin  本人の研究経費 ===================
	\KLMyBudget{}{}% 初年度と、期間全体で「本人が」使う額
%	分担者がいない場合は、\KLMyBudget{}{} のように{}の中を空にしてください。金額が自動的に入ります。
%end  本人の研究経費  ====================
}

\newcommand{\応募中の研究費}{%
%begin  応募中の研究費 ===================
		%6カラムのtabular
		% 1:資金制度・研究費名(研究期間・配分機関名)
		% 2:研究課題名(研究代表者氏名)
		% 3:代表/分担
		% 4:初年度の研究経費(期間全体の額)
		% 5:エフォート
		% 6:研究内容の相違点及び他の研究費に加えて本応募研究課題に応募する理由
%		\multicolumn{6}{c}{\dotfill} \\
		
		
%end  応募中の研究費 ====================
}

%====== end of page =====================================
%form: kiban_ab_form_16.tex ; user: kiban_ab_16_other_funds.tex
%========== S-1-7 基盤研究(A,B)(一般) =========
%===== p. 16 受け入れ予定の研究費 =============
\subsection{受け入れ予定の研究費}
%    \KLNoMarginMinipage{\KLLeftEdge}{694}{568}{
\newcommand{\受け入れ予定の研究費}{%
%begin  受け入れ予定の研究費 ===================
%		\multicolumn{6}{c}{\dotfill} \\
%end  受け入れ予定の研究費 ====================
}

%====== end of page =====================================
%\KLCheckPageLimit
%\KLAdvancePages
% hook9 : right before \end{document} ============

%endUserFiles
\input{forms/hook7} % for future maintenance

% kiban_ab_forms
%=======================================
\ifthenelse{\boolean{BudgetSummary}}{
	\KLTypesetPage{13}
	\KLTypesetPage{14}
}{}
	
\ifthenelse{\boolean{BudgetSummary}\OR \boolean{klTypesetPage0}}{
	\input{forms/coverpage}
}{}

\KLInputIfPageInRangeIsSelected{1}{2}{forms/kiban_ab_form_01-02}
\KLInputIfPageInRangeIsSelected{3}{5}{forms/kiban_ab_form_03-05}
\KLInputIfSelected{6}{forms/kiban_ab_form_06}
\KLInputIfPageInRangeIsSelected{7}{9}{forms/kiban_ab_form_07-09}
\KLInputIfSelected{10}{forms/kiban_ab_form_10}
\KLInputIfSelected{11}{forms/kiban_ab_form_11}
\KLInputIfSelected{12}{forms/kiban_ab_form_12}
\KLInputIfSelected{13}{forms/kiban_ab_form_13}
\KLInputIfSelected{14}{forms/kiban_ab_form_14}

\ifthenelse{\boolean{BudgetSummary}}{
	\input{forms/form90_summary}
}{%
}

\KLInputIfSelected{15}{forms/kiban_ab_form_15}
\KLInputIfSelected{16}{forms/kiban_ab_form_16}

%========================================

%endFormatFile

\input{forms/hook9} % for future maintenance
\end{document}
