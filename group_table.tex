\phantom{x}	\vspace{1cm}
%======================================
% group_table.tex (研究組織表)
%	2006-09-12 TakuYamanaka (Osaka Univ.)
%======================================
{\Large このファイルは、group\_table.tex です。}\\
研究組織表\\
\begin{tabular}{|l|l|l|r|p{1.5cm}|}
	\hline
	\KLGname{(研究者番号)}{(フリガナ)}{(漢字等)}{(年齢)}
	\KLGposition{(所属研究機関)}{(部局)}{職}
	\KLGfield{現在の専門}{学位}{役割分担}
	\begin{tabular}{l}
		初年度\\研究経費\\(千円)\\ 
		= \NumC{KLAnnualSum1}
	\end{tabular}
	& 
	エフォート(\%)\\
	\hline
	\hline
	
%===== 研究代表者 ==========================
	\multicolumn{5}{|l|}{研究代表者}\\
	\hline
	\KLGname{80398643}{チュウバチ ヨシヒデ}{\研究代表者氏名}{41}
			% 研究者番号/フリガナ/漢字等/年齢
	\KLGposition{\研究機関名}{産業技術大学院大学}{准教授}
			% 所属研究機関/部局/職
	\KLGfield{情報工学}{博士(政策・メディア)}{代表}
			% 現在の専門/学位/役割分担
	\KLGbudget{4400}
			% 初年度の研究経費(千円)[半角数字] 
	\本応募effort	% DO NOT TOUCH
	\\
	\hline
	\multicolumn{5}{|l|}{研究分担者}\\
	\hline
%===== 研究分担者 (例にならって、並べてください) =================
	
%============================================
	\hline
	\hline
	\multicolumn{2}{|c|}{合計 \arabic{KLNumPeople} 名} &
	研究経費合計 &
	\NumC{KLDistBudgetSum} & \\
	\hline
	\multicolumn{3}{|r|}{初年度に要求している予算額} &
	\NumC{KLAnnualSum1} & \\
	\hline
\end{tabular}

\ifthenelse{\value{KLAnnualSum1} = \value{KLDistBudgetSum}}{%
	OK : 研究代表者と分担者に配分した研究経費の合計は
	初年度の研究経費と一致しました。
}{
	{\LARGE ERROR: 研究代表者と分担者に配分した研究経費の合計が、\\
	初年度の研究経費\NumC{KLAnnualSum1} 千円と一致しません。}
}

%\end{document}  